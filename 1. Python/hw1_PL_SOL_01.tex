\documentclass{article}
\usepackage[utf8]{inputenc}
\usepackage{kotex}


\usepackage{natbib}
\usepackage{graphicx}

\begin{document}

\title{Binary Search in Python}
\author{B711222 박조은}
\date{April 2019}
\maketitle

\section{Introduction of the Problem}
[1, 11, 15, 19, 37, 48, 59, 61] \\
숫자 하나를 입력받아 위와 같은 순서의 리스트에서 입력한 숫자가 몇 번째 순서에 있는지를 Binary Search로 찾아 출력하는 함수를 구현하라. \\
(위 리스트 중 입력한 숫자가 없을 경우 None을 출력)

\section{What is Binary Search?}
이진 탐색이란 데이터가 정렬 돼있는 배열에서의 특정한 값을 찾아내는 알고리즘이다. \\
배열의 중간에 있는 임의의 값을 선택하여 찾고자 하는 값 X와 비교한다. \\
X가 중간 값보다 작으면 중간 값을 기준으로 좌측의 데이터들을 대상으로,\\
X가 중간값보다 크면 배열의 우측을 대상으로 다시 탐색한다. \\
동일한 방법으로 다시 중간의 값을 임의로 선택하고 비교한다. \\
해당 값을 찾을 때까지 이 과정을 반복한다.

\section{Explanation of the Code}
오름차순으로 정렬된 배열이 있다. \\
[1, 11, 15, 19, 37, 48, 59, 61] \\
이 배열에서 이진 탐색을 이용하여 15의 값을 찾아보자.\\
\\
첫 번째 시도\\
우선 가운데에 위치한 임의의 값 19를 선택한다. \\
선택한 값 19와 찾고자 하는 값 15를 비교한다. \\
15 < 19 이므로 15는 19의 좌측에 존재한다는 것을 알 수 있다. \\
\\
두 번째 시도\\
19를 기준으로 좌측에 있는 배열 값들을 대상으로 다시 탐색을 진행한다. \\
{1, 11, 15} \\
마찬가지로 가운데의 임의의 값 11을 선택한다. \\
11 < 15 이므로 이번에는 15가 11의 우측에 위치한다. \\
\\
세 번째 시도\\
11의 우측을 기준으로 배열을 다시 설정해보면 \\
{15}\\
배열에 값이 하나만 남게 되고 값을 확인해보면 \\
15 == 15 원하는 값을 찾았다.\\


\end{document}
