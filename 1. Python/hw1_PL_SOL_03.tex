\documentclass{article}
\usepackage[utf8]{inputenc}
\usepackage{kotex}


\usepackage{natbib}
\usepackage{graphicx}

\begin{document}

\title{Binary Search in Python}
\author{B711222 박조은}
\date{April 2019}
\maketitle

\section{Introduction of the Problem}
[100, 23, 31, 123, 435, 642, 1]\\
위와 같은 순서의 리스트가 있을 때 Merge Sort로 정렬하라.

\section{What is Merge Sort?}
합병 정렬은 다음과 같이 작동한다.\\
\\
리스트의 길이가 0 또는 1이면 이미 정렬된 것으로 본다. 그렇지 않은 경우에는\\
정렬되지 않은 리스트를 절반으로 잘라 비슷한 크기의 두 부분 리스트로 나눈다.\\
각 부분 리스트를 재귀적으로 합병 정렬을 이용해 정렬한다.\\
두 부분 리스트를 다시 하나의 정렬된 리스트로 합병한다.\\

\section{Explanation of the Code}
left: 100 / right:23 //
merge: 23 100 //
left: 31 / right: 123 //
merge: 31 123 //
left: 435 / right: 642 1 //
merge: 1 435 642 //
 //
left: 23 100 / right: 31 123 //
merge: 23 31 100 123 //
left: 23 31 100 123 / right: 1 435 642 //
merge: 1 23 31 100 123 435 642 //
\end{document}
