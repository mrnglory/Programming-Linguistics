\documentclass{article}
\usepackage[utf8]{inputenc}
\usepackage{kotex}


\usepackage{natbib}
\usepackage{graphicx}

\begin{document}

\title{Binary Search in Python}
\author{B711222 박조은}
\date{April 2019}
\maketitle

\section{Introduction of the Problem}
[26, 5, 37, 1, 61, 11, 59, 15, 48, 19] \\
위와 같은 순서의 리스트가 있을 때, Quick Sort로 정렬하라,

\section{What is Quick Sort?}
퀵정렬은 임의의 pivot 값을 기준으로 pivot 의 좌측에는 pivot 보다 작은값을 두고 우측에는 pivot 보다 큰 값을 두고자 한다.\\
\\
이 행위는 pivot을 기준으로 좌 우로 이분화 된 리스트를 재귀적으로 반복했을 때 결국 정렬이 완성 된다는 방법 론이다.\\
\\
보다 쉽게 설명하면 pivot보다 큰 값을 pivot index 보다 왼쪽에서 찾고 ( 큰 값 이 나타날 때까지 i index 를 증가시키도록 한다.)\\
\\
pivot 보다 작은 값을 pivot index 보다 오른쪽에서 찾는다 ( 작은 값이 나타날 때까지 j index를 감소시키도록 한다. )\\
\\
pivot을 기준으로 값 비교가 완료되었다면 index 결과 i , j 를 비교 해본다.\\
\\
i 값이 j 값 보다 작거나 같다면 분명 pivot 을 기준으로 교환을 해야할 값이 있다는 뜻이 된다.\\
\\
교환한 뒤 i 인덱스는 증가 j 인덱스는 감소 연산을 수행한다.\\
\\
i 인덱스가 j 인덱스보다 작거나 같다면 계속 반복해서 수행한다.\\
\\
위 와 같은 과정은 pivot을 기준으로 왼쪽으로 정렬된 list 에서는 최초 Left 값이 감소된 j 보다 작다면 계속 재귀하면되고,\\
\\
pivot을 기준으로 오른쪽으로 정렬된 list 에서는 최초 Right 값이 증가된 i 값보다 크다면 계속 재귀하면된다.\\
\\
\end{document}
