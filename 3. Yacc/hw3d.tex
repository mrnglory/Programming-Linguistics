\documentclass{article}
\usepackage[utf8]{inputenc}
\usepackage[hangul]{kotex}
\usepackage{bookman}
\usepackage{indentfirst}
\linespread{1.35}
\usepackage{listings}
\usepackage{graphicx}
\usepackage{enumitem}
\usepackage{listings}


\begin{document}

\title{Lex \& Yacc}
\author{Joeun Park}
\date{May 16th, 20}
\maketitle


\section{yacc에 대한 설명}
\subsection{yacc 개요}
yacc는 입력값에 대해 원하는 것을 찾아내는 일과, 그 찾아낸 것들 간의 관계를 따지는 프로그램 작성을 도와준다. 입력값을 의미단위(token)로 나누는 것을 어휘분석(Lexical Analysis)이라 하며, 그런 일을 하는 것을 어휘분석기(Lexical Analyzer or Lexer scanner)라고 한다. 입력값이 의미단위로 나뉘게 되면 프로그램은 그것들간의 관계를 따지게 된다. 예를 들어 C 컴파일러는 토큰의 수식, 문장, 선언문, 블록, 프로시저, 등과 같은 여부를 판별해낸다. 이런 작업이 구문분석이고 그 관계를 정의하는 규칙을 문법(Grammar)이라고 한다. \\\\
yacc는 구문 분석기(Parser)를 생성해주는 도구이다. 구문 분석기는 어휘 분석기로부터 받은 의미단위를 문법에 맞는지 검사하는 일을 한다. 사용자가 정의하고자 하는 문법을 C 언어로 기술해 주면 yacc가 자동적으로 구문 분석기를 생성해낸다. 구문 분석기는 의미단위를 가져와서 구문을 분석하는 프로그램이므로 어휘 분석을 하는 프로그램은 Lex로 별도로 구현시켜 두 프로그램을 접목시킨다.\\

\subsection{yacc 문법}
Yacc의 grammar file은 Lex Specification과 비슷하다.

\begin{lstlisting}
    %{
    C Definition Section
    %}
    
    Yacc Definition Section
    
    %%
    Grammar Rules Section
    %%
    
    User Subroutines
\end{lstlisting}

\begin{itemize}
    \item C Definition Section\\ 이 곳에 쓰여진 내용은 y.tab.c에 그대로 복사된다. Lex에서와 마찬가지로 다른 부분에서 사용할 변수가 있다면 이 곳에서 미리 정의하고 헤더파일을 include 해야한다.
    \item Yacc Definition Section\\ Grammar Rules Section에서 사용하는 토큰, 결합법칙, 변수 및 토큰들의 타입 등을 선언한다.
    \item Grammar Rules Section\\ 인식 하려는 문법과 그에 따른 C 언어 규칙을 정의한다. seperator는 :와 |이 쓰이며 terminator로는 ;이 쓰인다.
    \\item User Subroutines \\ 사용자가 만들어서 사용해야 할 함수가 있으면 이곳에 정의한다. 예를 들어 main()함수나 yyerror()등과 같은 함수들이 있다.
\end{itemize}
\par

\subsection{yacc 동작 방식}
Lex 혼자서도 어느정도 구문 분석을 하여 수행할 수 있다. Lex 에 의하여 만들어진 lexer 자신도 얼마든지 yacc 없이 특정한 문장을 읽어 우리가 원하는 동작을 구현할 수도 있다. lexer 에서 토큰이 하나하나 나올때 마다 특정한 행동을 수행하도록 하면 된다. 그러나 계산기와 같은 동작을 수행할 때 우리는 산술연산의 우선순위같은것을 정해 줘야 한다. 이런 c 코드를 만들기 위해서는 많은 노력이 필요하다. 즉 lexer에 의하여 나오는 토큰들을 수집하여, 수집된 토큰들을 나름대로 정리하여 원하는 계산 혹은 동작을 수행해야 한다. 이러기 위해서는 앞에서 언급했듯이 많은 노력이 필요로 하다. 이러한 노력을 줄이기 위하여 yacc를 사용하는 것이다.\\\\
yacc는 토큰을 계속 읽어서 사용자가 정의한 문법과 맞춰가며 구문 분석을 하도록 되어있다. 읽어들인 토큰이 문법을 온전히 만족하지 못하고 다른 토큰이 더 필요하다면 그 토큰을 스택에 쌓아 두는데 이것을 스택에 shift하며, 이를 shift라고 칭한다.\\\\
계속 토큰을 읽어서 스택에 있는 토큰들과 문법을 비교했을 때 문법을 만족하면 스택에 있던 토큰을 꺼내고(pop) 문법의 왼쪽(LHS) 심볼로 대치하는데, 이것을 reduce라 한다.\\\\
하지만 아무리 yacc가 구문분석을 하는데에 도움을 주는 툴이라 하더라도 하나의 파스 트리가 생성되지 못하는 모호한 문법 혹은 한 개를 초과하는 갯수의 토큰을 가져와야 분석을 할 수 있는 문법 등을 구문분석할 수는 없다.\\\\


\subsection{Lex 코드 분석}
\begin{lstlisting}

D			[0-9]
L			[a-zA-Z_]
H			[a-fA-F0-9]
E			[Ee][+-]?{D}+
FS			(f|F|l|L)
IS			(u|U|l|L)*

%{
#include <stdio.h>
#include "y.tab.h"
void count();
%}

%%
"/*"			{ comment(); }
"auto"			{ count(); return(AUTO); }
"break"			{ count(); return(BREAK); }
"case"			{ count(); return(CASE); }
"char"			{ count(); return(CHAR); }
"const"			{ count(); return(CONST); }
"continue"		{ count(); return(CONTINUE); }
"default"		{ count(); return(DEFAULT); }
"do"			{ count(); return(DO); }
"double"		{ count(); return(DOUBLE); }
"enum"			{ count(); return(ENUM); }
"extern"		{ count(); return(EXTERN); }
"float"			{ count(); return(FLOAT); }
"for"			{ count(); return(FOR); }
"goto"			{ count(); return(GOTO); }
"if"			{ count(); return(IF); }
"int"			{ count(); return(INT); }
"long"			{ count(); return(LONG); }
"register"		{ count(); return(REGISTER); }
"return"		{ count(); return(RETURN); }
"short"			{ count(); return(SHORT); }
"signed"		{ count(); return(SIGNED); }
"sizeof"		{ count(); return(SIZEOF); }
"static"		{ count(); return(STATIC); }
"struct"		{ count(); return(STRUCT); }
"switch"		{ count(); return(SWITCH); }
"typedef"		{ count(); return(TYPEDEF); }
"union"			{ count(); return(UNION); }
"unsigned"		{ count(); return(UNSIGNED); }
"void"			{ count(); return(VOID); }
"volatile"		{ count(); return(VOLATILE); }
"while"			{ count(); return(WHILE); }
{L}({L}|{D})*		{ count(); return(check_type()); }

0[xX]{H}+{IS}?		{ count(); return(CONSTANT); }
0{D}+{IS}?		{ count(); return(CONSTANT); }
{D}+{IS}?		{ count(); return(CONSTANT); }
L?'(\\.|[^\\'])+'	{ count(); return(CONSTANT); }
{D}+{E}{FS}?		{ count(); return(CONSTANT); }
{D}*"."{D}+({E})?{FS}?	{ count(); return(CONSTANT); }
{D}+"."{D}*({E})?{FS}?	{ count(); return(CONSTANT); }

L?\"(\\.|[^\\"])*\"	{ count(); return(STRING_LITERAL); }
"..."			{ count(); return(ELLIPSIS); }
">>="			{ count(); return(RIGHT_ASSIGN); }
"<<="			{ count(); return(LEFT_ASSIGN); }
"+="			{ count(); return(ADD_ASSIGN); }
"-="			{ count(); return(SUB_ASSIGN); }
"*="			{ count(); return(MUL_ASSIGN); }
"/="			{ count(); return(DIV_ASSIGN); }
"%="			{ count(); return(MOD_ASSIGN); }
"&="			{ count(); return(AND_ASSIGN); }
"^="			{ count(); return(XOR_ASSIGN); }
"|="			{ count(); return(OR_ASSIGN); }
">>"			{ count(); return(RIGHT_OP); }
"<<"			{ count(); return(LEFT_OP); }
"++"			{ count(); return(INC_OP); }
"--"			{ count(); return(DEC_OP); }
"->"			{ count(); return(PTR_OP); }
"&&"			{ count(); return(AND_OP); }
"||"			{ count(); return(OR_OP); }
"<="			{ count(); return(LE_OP); }
">="			{ count(); return(GE_OP); }
"=="			{ count(); return(EQ_OP); }
"!="			{ count(); return(NE_OP); }
";"			{ count(); return(';'); }
("{"|"<%")		{ count(); return('{'); }
("}"|"%>")		{ count(); return('}'); }
","			{ count(); return(','); }
":"			{ count(); return(':'); }
"="			{ count(); return('='); }
"("			{ count(); return('('); }
")"			{ count(); return(')'); }
("["|"<:")		{ count(); return('['); }
("]"|":>")		{ count(); return(']'); }
"."			{ count(); return('.'); }
"&"			{ count(); return('&'); }
"!"			{ count(); return('!'); }
"~"			{ count(); return('~'); }
"-"			{ count(); return('-'); }
"+"			{ count(); return('+'); }
"*"			{ count(); return('*'); }
"/"			{ count(); return('/'); }
"%"			{ count(); return('%'); }
"<"			{ count(); return('<'); }
">"			{ count(); return('>'); }
"^"			{ count(); return('^'); }
"|"			{ count(); return('|'); }
"?"			{ count(); return('?'); }
"#"			{ count(); return('#'); }

[ \t\v\n\f]		{ count(); }
.			{ count(); return('.'); }
%%


yywrap()
{
	return(1);
}

comment()
{
	char c, c1;
loop:
	while ((c = input()) != '*' && c != 0)
		putchar(c);

	if ((c1 = input()) != '/' && c != 0)
	{
		unput(c1);
		goto loop;
	}
	if (c != 0)
		putchar(c1);
}

int column = 0;

void count()
{
	int i;
	for (i = 0; yytext[i] != '\0'; i++)
		if (yytext[i] == '\n')
			column = 0;
		else if (yytext[i] == '\t')
			column += 8 - (column % 8);
		else
			column++;

	//ECHO;
}

int check_type()
{
/*
* pseudo code --- this is what it should check
*
*	if (yytext == type_name)
*		return(TYPE_NAME);
*
*	return(IDENTIFIER);
*/

/*
*	it actually will only return IDENTIFIER
*/

	return(IDENTIFIER);
}

\end{lstlisting}

\section{Yacc}
\subsection{yacc 전체 코드}
\begin{figure}[h]
\centering 
\includegraphics[width=\textwidth]{1.PNG} 
\end{figure}
\begin{figure}
\includegraphics[width=\textwidth]{2.PNG}  
\includegraphics[width=\textwidth]{3.PNG} 
\end{figure}
\begin{figure} [h]
\centering 
\includegraphics[width=\textwidth]{4.PNG}  
\includegraphics[width=\textwidth]{5.PNG}
\end{figure}
\begin{figure} [h]
\centering  
\includegraphics[width=\textwidth]{6.PNG}  
\includegraphics[width=\textwidth]{7.PNG} 
\end{figure}
\begin{figure} [h]
\centering 
\includegraphics[width=\textwidth]{8.PNG}  
\includegraphics[width=\textwidth]{9.PNG} 
\end{figure}
\begin{figure} [h]
\centering 
\includegraphics[width=\textwidth]{10.PNG}  
\includegraphics[width=\textwidth]{11.PNG} 
\end{figure}
\begin{figure} [h]
\centering 
\includegraphics[width=\textwidth]{12.PNG}  
\end{figure}

\subsection{yacc 요구사항과 미진한 부분 및 개선방안}
\begin{enumerate}
    \item 함수 카운팅\\전반적인 함수 카운팅은 구현하였으나, 포인터함수의 경우에서 함수로서 카운팅되지 않고 포인터변수로서 카운팅이 되며, 일반 함수 호출의 경우에 함수로서 카운팅이 되는 상황이다. 전자의 경우, checkpointer 변수와 checkfunc 변수가 둘 다 1을 만족하는 경우를 따져 이에 걸맞은 코드를 작성해야 했으나 아이디어가 부족했다. 후자의 경우, 함수가 {} 안에 코드를 품고있지 않은채 바로 ; 으로 끝나는 경우에 countfunc 변수를 0으로 하면 되지 않았나 싶다.
    \item 수식 카운팅\\연산, 비트연산, 비교, 포인터기호 등과 같은 operator 카운팅을 구현하였다.
    \item int 변수 카운팅\\int 변수의 전반적인 카운팅은 구현하였으나, int main(){} 과 같은 int형 함수 혹은 int형 포인터함수 등의 경우는 int형 변수로 카운팅이 들어가지 않아야 하는데, 이러한 경우까지 고려하는 코드를 작성 하는 아이디어가 부족했다. \\ 이 부분을 개선하기 위해 함수가 카운팅 되는 구간에 intcount 변수를 0으로 만들거나 감소시켜야 한다고 생각한다.
    \item char 변수 카운팅\\char 변수의 전반적인 카운팅은 구현하였으나, int 변수 카운팅의 상황과 마찬가지의 오류를 범할 수 있다.
    \item pointer 변수 카운팅\\위에도 잠깐 언급하였듯이, 포인터함수의 경우에는 pointer 변수 카운팅이 들어가면 안 되는데, 이 오류를 바로잡는 데에 실패하였다. \\ 이 부분을 개선하기 위해서는 int *a(){} 와 같이 뒤에 '(', ')', '{', '}'이 오는 경우의 함수 조건에 pointercount 변수를 0으로 설정하거나 감소시켜야 한다고 생각한다.
    \item 배열 변수 카운팅\\일차원 이상의 차원 배열 변수는 모두 배열변수가 1개인 경우이므로 이 조건을 잘 고려하여 구현하였다.
    \item 선택문 카운팅\\선택문이 카운팅 되도록 잘 구현하였으며, 포함시키지 않아도 되는 조건의 경우는 제거하였다.
    \item 반복문 카운팅\\반복문이 카운팅 되도록 잘 구현하였다.
    \item 리턴문 카운팅\\리턴문이 카운팅 되도록 잘 구현하였다.
    \item 주의사항\\#include와 같은 헤더파일이나 #define문을 갖는 c 코드에서 오류가 생기지 않도록 yacc 문법을 잘 수정하였다.\\테스트 케이스에 있을 수 있다고 언급된 타입들을 토큰화 시켰다. \\ 주의사항 6번부터 10번까지의 경우에 대해서는 혼자서 c 코드 경우를 구현 하지 못했기 때문에 깊이 생각해보지 못했다.
    \item 기타사항\\추가 공지사항에 의하면 printf(" ' ");, printf(' " '); 와 같은 경우도 생각해줘야 한다고 했는데, c 코드를 돌려볼 때 이에 관련된 오류를 발견하지 못했고 그 밖의 설명이나 문제거리를 접한 바가 없어서 무슨 경우를 뜻하는 것인지 확신하지 못한 상황이다.\\또한, LaTeX에서의 Lex 코드 설명란에서 주석을 LaTeX로 구현하는 것을 숙지하지 못하였다. 
\end{enumerate}

\section{Reference}
\begin{enumerate}
    \item https://stackoverflow.com/questions/41900370/writing-lex-and-yacc-rules-to-detect-include-in-c-language
    \item https://kgon.tistory.com/90
    \item https://github.com/SilverScar/C-Language-Parser/commit/f27512f6c789481ad87aaf0c32d5300f07f138e0#diff-b6422c90c0c2b8118486a76df593ce14R13
    \item https://stackoverflow.com/questions/56087456/is-there-any-c-library-for-drawing-pictures-and-making-jpg-files
    \item https://www.lysator.liu.se/c/ANSI-C-grammar-l.html
    \item https://www.lysator.liu.se/c/ANSI-C-grammar-y.html
    \item https://stackoverflow.com/questions/41900370/writing-lex-and-yacc-rules-to-detect-include-in-c-language
    \item https://www.joinc.co.kr/w/Site/Development/Env/Yacc

\end{enumerate}
\end{document}
